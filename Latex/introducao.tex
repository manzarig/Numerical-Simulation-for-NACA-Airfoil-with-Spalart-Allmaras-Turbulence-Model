
\chapter{Introdução}
\label{cap:cap01}

Desde o século XIX, a humanidade utiliza o petróleo em diversas áreas. 
Atualmente, este possui grande participação na economia global, 
principalmente como combustível para veículos com motores a combustão interna. 
O petróleo, além de ser um recurso não-renovável, é utilizado na produção da gasolina e do óleo diesel, que, por sua vez, 
é responsável por 72,6\% das emissões de gases do efeito estufa. 
Com isso, fica evidente a problemática gerada pelo uso descontrolado do petróleo como fonte de energia: 
um grande aumento no efeito estufa. A procura de uma solução para tal problema se iniciou mais de um século atrás, e, 
hodiernamente, os e-fuels (combustíveis sintéticos) são comumente apontados como o resultado dessa busca.

Os combustíveis sintéticos, diferentemente daqueles derivados do petróleo, são renováveis, já que sua matéria prima é, como sugere o nome, sintética. Ademais, sua produção não libera CO2, o que ajuda na resolução do problema do aquecimento global. No entanto, esses combustíveis também possuem pontos negativos, como o fato de a sua eficiência energética ser baixa, e o seu custo de produção ser elevado.

Tendo isso em vista, este trabalho discutirá a viabilidade econômica do uso desses combustíveis como substitutos daqueles derivados do petróleo. Para tal, analisar-se-ão diferentes cenários de aplicação, como para uso da população geral, para uso no automobilismo, e em comparação com carros elétricos.